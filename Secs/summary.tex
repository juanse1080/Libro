% ------------------------------------------------------------------------
% ------------------------------------------------------------------------
% ------------------------------------------------------------------------
%                                Resumen
% ------------------------------------------------------------------------
% ------------------------------------------------------------------------
% ------------------------------------------------------------------------

\chapter*{RESUMEN}
\footnotesize{
\begin{description}
  
  \item[TÍTULO:] DISEÑO DE UNA PLATAFORMA SOFTWARE EXTENSIBLE PARA EL DESPLIEGUE, EJECUCIÓN Y VISUALIZACIÓN DE ALGORITHMOS DEDICADOS A LA VISIÓN POR COMPUTADOR \astfootnote{Trabajo de grado}
  
  \item[AUTORES:]
  JUAN SEBASTIAN MARCON CABALLERO
  \asttfootnote{Facultad de Ingenierías Físico-Mecánicas. Escuela de Ingeniería de Sistemas e Informática. Director: Fabio Martínez Carrillo, Ph.D. Codirector: Gabriel Rodrigo Pedraza Dr.}
  
  \item[PALABRAS CLAVE:] ALGORITMOS DEDICADOS, VISIÓN POR COMPUTADOR, APRENDIZAJE DE MÁQUINA, DESPLIEGUE, EJECUCIÓN, VISUALIZACIÓN, COMPATIBILIDAD, EXTENSIBLE, ESTRUCTURA DE SOFTWARE.
  
  \item[DESCRIPCIÓN:] 
    El auge hoy en día, en áreas como la visión por computador e inteligencia artificial han permitido soportar y dar solución a un gran número de problemas en diversas áreas del conocimiento. Muchos de estos métodos han evolucionado a aplicaciones robustas que operan en el mercado con éxito y dan solución a problemas relacionados con el análisis de datos y la simplificación de tareas repetitivas.  Sin embargo, algunas propuestas de investigación orientadas al diseño de estos algoritmos no evolucionan como aplicaciones tecnológicas debido a la dificultad existente en el proceso de despliegue y ejecución de los diferentes modelos que los componen, requiriendo así el diseño de software que ofrezca una infraestructura adecuada para el despliegue, la administración de la complejidad, el manejo de la heterogeneidad y la extensibilidad de la solución. \\
    Las herramientas características de las ciencias de la computación están centradas en la solución de problemas de optimización y modelamiento de algoritmos de aprendizaje de máquina. Sin embargo estos algoritmos carecen de una estructura de software clara y uniforme, lo que conlleva a prototipos funcionales, pero con poca usabilidad, inclusive por parte de otros expertos en la misma área. Entre otras, estos algoritmos funcionales tienen múltiples dependencias de software que generan problemas de incompatibilidad y dificultad de operar en diferentes ambientes y sistemas operativos. Por lo anteriormente expuesto, en este proyecto se pretende diseñar una plataforma de software que permita utilizar, visualizar y desplegar, diferentes tecnologías basadas en el aprendizaje de máquina y la visión por computador. \\



\end{description}
}\normalsize
% ------------------------------------------------------------------------ 