% ------------------------------------------------------------------------
% ------------------------------------------------------------------------
% ------------------------------------------------------------------------
%                              Introducción
% ------------------------------------------------------------------------
% ------------------------------------------------------------------------
% ------------------------------------------------------------------------

\nnchapter{INTRODUCCIÓN}

Las nuevas tecnologías se acoplan de forma natural, a tal punto que la interacción con estas resulta normal en la cotidianidad. Hoy en día, las aplicaciones ofrecen diferentes funcionalidades que resultan esenciales para el usuario, lo cual resulta en un prolongado uso de la aplicación. Gran parte de su porcentaje de persistencia se logra mediante un valor agregado que brinde un aporte diferencial que resalte la aplicación de otra con el mismo objetivo. El valor agregado o factor adicional en sistemas informáticos en muchos casos ha sido implementado usando ramas de la inteligencia artificial. El aporte de estas ramas va mucho más allá de brindar agentes diferenciadores, estas han permitido soportar y dar solución a un sin número de problemas en diversas áreas del conocimiento y han sido usadas en un amplio número de campos \myfootcite{rahwan_2009}, incluyendo diagnóstico médico, comercio de acciones, control robótico, aviación, exploración espacial, leyes, percepción remota, descubrimientos científicos, juegos, entre otros.\\


A pesar de su gran aplicabilidad y alta demanda, algunos algoritmos no entran a su etapa de producción debido a la complejidad que lleva el despliegue, además, muchos de estos algoritmos son basados en computación distribuida, lo cual requiere la factorización del código fuente \myfootcite{gropp1999using}. La solución a lo anterior es la implementación de software especializado para cada algoritmo y el entendimiento de la topología que implica un correcto entendimiento del algoritmo \myfootcite{fan_chen_wang_zheng_lyu_2014}. Sin embargo la creación de cada algoritmo tiene múltiples soluciones y múltiples tecnologías de desarrollo lo cual requiere una evaluación tecnológica \myfootcite{compatibilidad} de lo contrario puede llevar a problemas de compatibilidad entre estas tecnologías impidiendo la coexistencia de tecnologías incompatibles entre sí. Este fenómeno obliga a los desarrolladores a la invalidación de herramientas de desarrollo e impone a la comunidad la estandarización de tecnologías de desarrollo para este tipo de algoritmos.\\


Por las anteriores razones se pretende diseñar un prototipo de software basado en una arquitectura de microservicios que permita al usuario utilizar, visualizar y desplegar, diferentes algoritmos relacionados con el aprendizaje de máquina y la visión por computador. Este aplicativo pretende ofrecer flexibilidad de comunicación entre el núcleo y servicios mediante un framework de conexión, el cual debe permitir la definición de la comunicación entre estos. Además abordaremos los problemas de compatibilidad mediante el aislamiento de entornos de ejecución.

% ------------------------------------------------------------------------
% ------------------------------------------------------------------------
